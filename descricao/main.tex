\documentclass[11pt]{article}

% Pacotes necessários
\usepackage[utf8]{inputenc}
\usepackage{graphicx}
\usepackage{caption}
\usepackage{lipsum}
\usepackage{mathpazo}
\usepackage[a4paper, left = 2cm, right = 2cm, bottom = 2cm, top = 2cm]{geometry}
\usepackage[brazil]{babel}
\usepackage{float}
\usepackage{xcolor}
\usepackage{hyperref}
\usepackage{indentfirst}
\usepackage{caption}
\usepackage{url}
\usepackage{subfig}
\usepackage{url}
\usepackage{booktabs}
\usepackage{makecell, multirow, tabularx}
\usepackage{url}
\usepackage{blindtext}

\setlength{\parskip}{\baselineskip}
\setlength{\parindent}{17pt}

\usepackage[square,numbers]{natbib}

\hypersetup{
	colorlinks=true,       % false: boxed links; true: colored links
	linkcolor=purple,      % color of internal links (change box color with linkbordercolor)
	citecolor=black,       % color of links to bibliography
	filecolor=black,       % color of file links
	urlcolor=purple        % color of external lin
}

\title{
	Descrição do projeto de trabalho de Matemática Discreta \\ \vspace{0.2cm}
	\large \textit{Matchings} e ciclos Hamiltonianos em grafos de hipercubos
}

\author{
	Amanda de Mendonça Perez \\
	Juan Belieni de Castro Araujo
}

\date{\today}

\begin{document}
	
\maketitle

\section{Descrição}

% nesse trabalho pretendemos explorar ...
Neste trabalho, pretendemos explorar definições, teoremas e algoritmos envolvendo hipercubos, em especial associados a \textit{matchings} e ciclos Hamiltonianos. Primeiramente, vamos definir grafos de hipercubos e discutir algumas de suas propriedades, utilizando como base o trabalho de \citet{harary_survey_1988}. Depois, vamos apresentar o teorema que todo \textit{perfect matching} desse tipo de grafo pode ser estendido em um ciclo Hamiltoniano, nos baseando na prova de \citet{fink_perfect_2007}.

Se possível, pretendemos ainda implementar algum algoritmo para estender o ciclo Hamiltoniano a partir de um dado \textit{perfect matching}, para que seja possível visualizar o ciclo no grafo.

\bibliographystyle{unsrtnat}
\bibliography{refs}

\end{document}
